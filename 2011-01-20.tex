\section{VL von 20.~Januar 2011}

Kompaktheit Eigenschaft von FO.\\
SO ist nicht entscheidbar. Gültigkeit ist rekursiv aufzählbar in FO.\\
MSO charakterisiert formale Sprachen.\\
Temporallogik Fragment von FO.\\

\subsection{Beispiele zur Semantik}

\begin{align*}
  \exists z&.P(z) \AND Q(x) \\
  \exists X\exists z&.X(z) \AND X'(z) \\[\baselineskip]
  \Afrak &= (\N, <) \\
  \beta &= \set{x\mapsto 3,y\mapsto 4, X\mapsto\set{(4,3)}} \\
  \Afrak, \beta & \not\models^? X(x,y) \\
    &\text{da}\ (3,4) \not\in \beta(X) \\[\baselineskip]
  \Afrak, \set{x\mapsto 3,y\mapsto 4} &\models^! \exists X.X(x,y)\\
    &\text{z.B.}\ \beta(<) \\
    &\text{oder}\ \beta(X) = \set{(3,4)} \\[\baselineskip]
  \Afrak, \set{x\mapsto 3,y\mapsto 4} &\not\models \forall X.X(x,y) \\
    &\text{da}\ \beta(X) = > (\text{gilt nicht, da }(\beta(X) = >) \text{sein kann})
\end{align*}

\begin{verbatim}
  \Afrak
     1   E    2        3
      o-----> o <---- o
       \      |       |
         \    |       |
           \  |       |
             \|       |
              v       v
              o ----> o
             4         5
\end{verbatim}

Alle Nachfolger von 1 müssen drin sein. 3 kann drin sein, muss aber nicht.

\begin{align*}
  \Afrak &= \varphi(1,5)\\
  X &= \set{1,2,4,5} \\[\baselineskip]
  \Afrak &= \varphi(1,3)
\end{align*}

3 ist nicht drin. Das gint in FO nicht

\subsubsection{Beweis}

\begin{align*}
  \psi(X) &= \forall z,z'.(X(z) \AND E(z,z') \IMPL X(z'))\\
  \varphi(x,y) &= \forall X.(X(x) \AND \psi(X) \IMPL X(y))
\end{align*}

\begin{description}[style=nextline]
  \item[Behauptung:] $\Afrak\models\varphi[a,b]$ gdw. es gibt einen Pfad von $a$ nach $b$.
  
  \item[\enquote{$\Rightarrow$}]
  Definiere $R=\set{\hat{a}\in A| \text{es gibt einen Pfad von a nach b}}$.
  $\Afrak\models\psi[R]$ ist offensichtlich der Fall. Außerdem gilt $a\in R$.
  Da $\Afrak\models\varphi(a,b)$, gilt also $b\in R$. Per Definition gibt
  es einen Pfad von $a$ nach $b$.
  
  \item[\enquote{$\Leftarrow$}]
  Sei $R\subseteq A$ beliebig und $a\in R$ sowie $\psi(R)$. Zu zeigen: $b\in R$.
  Das folgt direkt, da $a\in R$ und alle Nachfolger von $a$ ebenfalls.
  Die Annahme war, dass es einen Pfad von $a$ nach $b$ gibt.
\end{description}

\subsection{Beispiel 2}

\begin{align*}
  \varphi &= \exists R.\Bigl(
    \forall x.\exists y.R(x,y) \OR R(y,x)
    \AND \func(R)
    \AND \func^-(R)
    \AND \forall x.\bigl(\exists y.R(x,y) \IMPL \NOT\exists y.R(y,x)\bigr)
  \Bigr)
\end{align*}
