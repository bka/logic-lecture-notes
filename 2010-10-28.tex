\section{VL vom 28.~Oktober 2010}

\subsection{Ersetzungslemma}

\begin{description}
  \item[IA:]
  Wenn $\vartheta$ atomar ist, muss $\vartheta = \varphi$. Dann $\vartheta'=\psi$,
  also $\vartheta\EQUIV\vartheta'$ wegen $\psi\EQUIV\varphi$.

  \item[IS:]
  Wenn $\vartheta=\varphi$, argumentiere wir wie im IA. Sonst unterscheide drei Fälle "uber den "au"seren Junktor:

  \begin{enumerate}
    \item $\vartheta = \NOT \vartheta_1$\\
    $\vartheta'$ hat die Form $\NOT\vartheta'_1$ (wobei sich $\vartheta'_1$ aus
    $\vartheta_1$ ergibt durch Ersetzen von $\varphi$ durch $\psi$. Nach IV gilt
    $\vartheta_1\EQUIV\vartheta'_1$, nach Semantik von \enquote{$\NOT$} also
    $\vartheta\EQUIV\vartheta'$.
    
    \item $\vartheta = \vartheta_1\OR\vartheta_2$\\
    $\varphi$ wird entweder in $\vartheta_1$ oder in $\vartheta_2$ durch $\psi$
    ersetzt. Wir betrachten nur den ersten Fall: dann
    $\vartheta'=\vartheta'_1\OR\vartheta_2$. Nach IV $\vartheta_1\EQUIV\vartheta'_1$, also $\vartheta\EQUIV\vartheta'$.
    
    \item $\vartheta = \vartheta_1\AND\vartheta_2$\\
    Analog zu 2.
  \end{enumerate}
  \qed
\end{description}


\subsection{Boolsche Funktionen}
$B^n$ hat $2^n$ m"ogliche Eingaben. Beispiel $B^2$:
\begin{tabular}{l|l|l|l}
Eingaben    & f1  & f2  & \dots \\
\hline
(0,0)       & 0   & 0   & \dots \\
(0,1)       & 0   & 0   & \dots \\
(1,0)       & 0   & 0   & \dots \\
(1,1)       & 0   & 1   & \dots
\end{tabular}

\subsection{Funktionale Vollständigkeit}

Die Funktionen in $\mathcal{B}^0$ werden durch die Formeln $0,1$ dargestellt.
Sei $n>0$ und $f\in\mathcal{B}^n$.

Für jedes $x\in Var$ sei $x^1=x$ und $x^0=\NOT x$. Für jedes Tupel (für jede Eingabe)
$f=(w_1,\dots,w_n)\in\set{0,1}^n$ sei $\varphi_f=x_1^{w_1}\AND\dots\AND x_n^{w_n}$.

\textbf{Definiere}
\begin{align}
  \varphi_f = \ORop_{\substack{f\in\set{0,1}^n\\ f(t) = 1}} \varphi_t
\end{align}

\textbf{Behauptung:} $f_{\varphi_f} = f$

Wenn $f(t) = 1$, dann ist $\varphi_t$ ein Disjunkt von $\varphi_f$. Also
$f_{\varphi_f}(t) = V_t(\varphi_f)=1$. Wenn umgekehrt $f_{\varphi_f}(t) = 1$,
dann $V_t(\varphi_f)=1$, also ist $\varphi_t$ Disjunkt von $\varphi_f$. Nach
Definition von $\varphi_f$ also $f(t)=1$. \qed


\subsection{Normalformen}

Sei $\varphi$ eine Formel. Äquivalente Formel in DNF ($\OR\AND$):

Konstruiere erst $f_\varphi$ mittels Wahrheitstafel, dann $\varphi_{f_\varphi}$
wie im vorigen Beweis. Das Resultat ist offensichtlich in DNF und die
Konstrktion ist effektiv.

Aus der effektiven Konstruierbarkeit der DNF folgt auch die der KNF ($\AND\OR$).

\begin{align}
  \varphi &\EQUIV \NOT\NOT \varphi                                 &&\text{DNF!}\\
          &\EQUIV \NOT\ORop_{i=1}^n \ANDop_{j=1}^{m_i} \ell_{i,j}  &&\text{de Morgan}\\
          &\EQUIV \ANDop_{i=1}^n \NOT\ANDop_{j=1}^{m_1} \ell_{i,j} &&\text{de Morgan}\\
          &\EQUIV \ANDop_{i=1}^n \ORop_{j=1}^{m_1} \NOT\ell_{i,j}  &&\text{KNF}
\end{align}
\qed


Paritätsfunktion Divide\&Conquer\\
Boolsche Funktionen bilden Wahrheitswerte auf Warheitswerte ab.\\
Zu jeder log. Funktion ich boolsche Funktion finden, die sie repräsentiert.\\

\seeslide{30}

Geringste Junktormenge: $\NOT, \AND$ oder $\NOT, \OR$ sind ausreichend\\

0,1 aus $B^0$\\
$\NOT$ aus $B^1$\\
$\OR,\AND$ aus $B^2$\\
Umgekehrt: jede boolsche Funktion liefert n-ären Junktor.

Gültigkeit: Reduktion auf Unerfüllbarkeit u.U.

Folgerbarkeit

Die Belegung v erfüllt $\varphi$ :\\
$v \models \varphi$ \\
Jede Belegung die x und y wahr macht macht auch x wahr:\\
$x \AND y \models y$\\

Jede Belegung, die $\varphi$ erfüllt, erfüllt auch $\vartheta$.\\
$\varphi \models \vartheta$.

Horn Formel KNF mit höchstens einem positiven Literal.\\
\seeslide{42}
Äquivalenzen prüfen.\\
